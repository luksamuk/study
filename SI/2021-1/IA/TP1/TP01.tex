% Created 2021-10-24 dom 03:31
% Intended LaTeX compiler: pdflatex
\documentclass[article, a4paper, oneside, 11pt, english, brazil, sumario=tradicional]{abntex2}
		  \usepackage{times}
\usepackage[utf8x]{inputenc}
\usepackage[T1]{fontenc}
\usepackage{titlesec}
\usepackage[english, hyperpageref]{backref}
\usepackage{hyperref}
\usepackage{indentfirst}
\usepackage{titling}
\ifthenelse{\equal{\ABNTEXisarticle}{true}}{%
\renewcommand{\maketitlehookb}{}
}{}
\titleformat{\section}{\normalfont\normalsize\bfseries\uppercase}{\thesection\space\space}{0pt}{}
\titleformat{\subsection}{\normalfont\normalsize\bfseries}{\thesubsection\space\space}{0pt}{\space}
\titleformat{\subsubsection}{\normalfont\normalsize\bfseries}{\thesubsubsection\space\space}{0pt}{\space}
\titleformat{\paragraph}{\normalfont\normalsize\itshape}{}{0pt}{\theparagraph\space\space}
\setlength{\parindent}{1.5cm}
\setlrmarginsandblock{3cm}{2cm}{*}
\setulmarginsandblock{2.5cm}{2.5cm}{*}
\checkandfixthelayout



\usepackage{minted}
\author{Lucas Samuel Vieira\thanks{lucas.vieira@ufvjm.edu.br}}
\date{\today}
\title{Inteligência Artificial - Trabalho Prático 01 - Respostas\\\medskip
\large Introdução à Linguagem de Programação Python}
\hypersetup{
 pdfauthor={Lucas Samuel Vieira},
 pdftitle={Inteligência Artificial - Trabalho Prático 01 - Respostas},
 pdfkeywords={},
 pdfsubject={},
 pdfcreator={Emacs 27.2 (Org mode 9.4.5)}, 
 pdflang={Brazilian}}
\begin{document}

\maketitle
\OnehalfSpacing
\pretextual
\textual

\section{Exercício 01 - Padaria}
\label{sec:org81f226d}

\begin{minted}[]{python}
from typing import Final

# O pão dormido tem 60% de desconto, ou seja,
# tem 40% do valor do pão fresco
BREAD_COST: Final = 4.60
OLD_BREAD_COST: Final = BREAD_COST * 0.4

def print_aligned(prompt, number):
    print('{}{:6.2f}'.format(prompt, number))

# Input do usuário
while True:
    try:
        old_breads = int(input("Insira a quantidade de paes dormidos: "))
        if old_breads < 0:
            print("Quantidade deve ser maior ou igual a zero.")
        else:
            break
    except ValueError:
        print("Quantidade invalida.")

# Exibição das informações
print_aligned('Preco do pao:         ', BREAD_COST)
print_aligned('Preco do pao dormido: ', OLD_BREAD_COST)
print_aligned('Total da compra:      ', old_breads * OLD_BREAD_COST)
\end{minted}

\newpage
\section{Exercício 02 - Roleta}
\label{sec:org412f213}

\begin{minted}[]{python}
from random import randint
from typing import Final

def gen_random_number():
    return randint(-1, 36)

# Constantes indicando cores das casas (-1 == 00).
GREEN_NUMBERS: Final = [-1, 0]
RED_NUMBERS: Final = [
    1,   3,  5,  7,  9, 12, 14, 16, 18,
    19, 21, 23, 25, 27, 30, 32, 34, 36
]

def get_number_color(number):
    if number in GREEN_NUMBERS:
        return "Verde"
    elif number in RED_NUMBERS:
        return "Vermelho"
    return "Preto"

def get_roulette_range(number):
    if number >= 1 and number <= 18:
        return "1 a 18"
    return "19 a 36"

def format_roulette_number(number):
    return "00" if number == -1 else "{}".format(number)

def get_payment_categories(number):
    categories = []
    # Número único
    categories.append(format_roulette_number(number))
    if number > 0:
        # Vermelho ou preto
        categories.append(get_number_color(number))
        # Ímpar ou par
        categories.append("par" if number % 2 == 0 else "ímpar")
        # Intervalo da roleta
        categories.append(get_roulette_range(number))
    return categories


generated_number = gen_random_number()
print("O resultado da rodada é {}"
      .format(format_roulette_number(generated_number)))

for category in get_payment_categories(generated_number):
    print("Pagar {}".format(category))
\end{minted}

\newpage
\section{Exercício 03 - Maior valor inteiro}
\label{sec:org818aa92}

\begin{minted}[]{python}
from random import randint

num_updates = 0
maximum = randint(1, 100)
print("{}".format(maximum))

for i in range(99):
    new_num = randint(1, 100)
    if new_num > maximum:
        maximum = new_num
        num_updates += 1
        print("{} (atualizado)".format(new_num))
    else:
        print("{}".format(new_num))

print("O valor máximo encontrado foi {}".format(maximum))
print("O número máximo de vezes que o maior valor foi "
      "atualizado foi {} vezes."
      .format(num_updates))
\end{minted}

\section{Exercício 04 - Ano bissexto}
\label{sec:orgdda02d4}

\begin{minted}[]{python}
def is_bissexto(year):
    if year % 400 == 0:
        return True
    elif year % 100 == 0:
        return False
    elif year % 4 == 0:
        return True
    return False

# Resultado deve ser [True, False, True, False]
years = [2016, 2018, 2020, 2021]
print(list(map(is_bissexto, years)))
\end{minted}

\section{Exercício 05 - Dias em um mês}
\label{sec:orgc43fa98}

\begin{minted}[]{python}
# Exercício 04
def is_bissexto(year):
    if year % 400 == 0:
        return True
    elif year % 100 == 0:
        return False
    elif year % 4 == 0:
        return True
    return False

def num_days(month, year):
    if month == 2:
        return 29 if is_bissexto(year) else 28
    elif month in [1, 3, 5, 7, 8, 10, 12]:
        return 31
    return 30

# Deverá imprimir [31, 28, 29, 29]
test_args = [(1, 2021), (2, 2021), (2, 2020), (2, 1996)]
print(list(map(lambda p: num_days(p[0], p[1]), test_args)))
\end{minted}

\newpage
\section{Exercício 06 - Data mágica}
\label{sec:org2533c67}

\begin{minted}[]{python}
# Exercício 04
def is_bissexto(year):
    if year % 400 == 0:
        return True
    elif year % 100 == 0:
        return False
    elif year % 4 == 0:
        return True
    return False

# Exercício 05
def num_days(month, year):
    if month == 2:
        return 29 if is_bissexto(year) else 28
    elif month in [1, 3, 5, 7, 8, 10, 12]:
        return 31
    return 30

def is_magic_date(day, month, year):
    year_two_digits = year - (int(year / 100) * 100)
    return (day * month) == year_two_digits

# Encontra todas as datas mágicas do Século XX (1901-2000).
for year in range(1901, 2001):
    for month in range(1, 13):
        for day in range(1, num_days(month, year)+1):
            if is_magic_date(day, month, year):
                print("{:02d}/{:02d}/{:04d}"
                      .format(day, month, year))
\end{minted}

\newpage
\section{Exercício 07 - Crivo de Eratóstenes}
\label{sec:org047bab5}

\begin{enumerate}
\item Ao invés de remover diretamente os  itens da lista, algo que seria inseguro e
lento, optei por marcá-los  como \texttt{False} em uma lista que  indica se o número
naquele índice é primo ou não.
\item Só é necessário gerar múltiplos se o  \(p\) atual for um número primo, por mais
que iteremos até o limite.
\item O  cálculo de múltiplos é  realizado para \(p^2 \leq  x \leq \textrm{limite}\),
através de um range com passos de \(p\) em \(p\), para evitar cálculos diretos de
multiplicação.
\end{enumerate}

\begin{minted}[]{python}
def sieve_of_erathostenes(limit):
    is_prime = [True for i in range(limit + 1)]
    is_prime[0] = False
    is_prime[1] = False
    for p in range(2, limit):
        if is_prime[p]:
            for multiple in range(p ** 2, limit + 1, p):
                is_prime[multiple] = False
    # Geração da lista
    primes = []
    for i in range(limit + 1):
        if is_prime[i]:
            primes.append(i)
    return primes

while True:
    try:
        limit = int(input("Insira um limite: "))
        if limit < 2:
            print("Insira um número maior ou igual a 2.")
            continue
        break
    except ValueError:
        print("Valor inválido.")

print(sieve_of_erathostenes(limit))
\end{minted}

\newpage
\section{Exercício 08 - Scrabble}
\label{sec:org66ea7c8}

\begin{minted}[]{python}
# Construção do dictionary
def build_dict():
    points = {}
    for letter in ['a', 'e', 'i', 'l', 'n', 'o', 'r', 's', 't', 'u']:
        points[letter] = 1
    for letter in ['d', 'g']:
        points[letter] = 2
    for letter in ['b', 'c', 'm', 'p']:
        points[letter] = 3
    for letter in ['f', 'h', 'v', 'w', 'y']:
        points[letter] = 4
    points['k'] = 5
    for letter in ['j', 'x']:
        points[letter] = 8
    for letter in ['q', 'z']:
        points[letter] = 10
    return points

points_dict = build_dict()
points = 0
word = input('Insira uma palavra: ').lower()

for letter in word:
    try:
        points += points_dict[letter]
    except KeyError:
        pass

print("Pontuação: {} pontos".format(points))
\end{minted}

\newpage
\section{Exercício 09 - Mini-parser}
\label{sec:org772dc64}

\begin{minted}[]{python}
#!/usr/bin/python
import sys

def slurp(filename):
    slurped_file = []
    try:
        with open(filename, "r") as the_file:
            slurped_file = [line.rstrip('\n') for line in the_file]
    except IsADirectoryError:
        print("Erro ao abrir \"{}\": é um diretório".format(filename))
    except FileNotFoundError:
        print("Erro ao abrir o arquivo \"{}\"".format(filename))
    except OSError:
        print("Impossível ler o arquivo \"{}\"".format(filename))
    return slurped_file

def get_function_name(line):
    return line.split(" ")[1].split("(")[0]

def print_undocumented_function(filename, line, function_name):
    print("{}:{:03d}: {}".format(filename, line, function_name))

def check_file(filename):
    lines = slurp(filename)
    if len(lines) > 0:
        # Caso especial para primeira linha: se for uma
        # função, não tem comentário antes.
        if lines[0].startswith("def "):
            print_undocumented_function(
                filename, 1, get_function_name(lines[0]))
        for idx in range(1, len(lines)):
            prev_line = lines[idx - 1]
            curr_line = lines[idx]
            if curr_line.startswith("def ") and \
               (not prev_line.startswith("#")):
                print_undocumented_function(
                    filename, idx + 1, get_function_name(curr_line))

for arg in sys.argv[1:]:
    check_file(arg)
\end{minted}

\newpage
\section{Exercício 10 - Distância entre Strings}
\label{sec:org71a3252}

\begin{minted}[]{python}
def string_distance(s, t):
    if len(s) == 0:
        return len(t)
    elif len(t) == 0:
        return len(s)
    cost = 0
    if s[-1] != t[-1]:
        cost = 1
    d1 = string_distance(s[:-1], t) + 1
    d2 = string_distance(s, t[:-1]) + 1
    d3 = string_distance(s[:-1], t[:-1]) + cost
    return min(d1, d2, d3)

str1 = input("Insira um texto:    ").lower()
str2 = input("Insira outro texto: ").lower()
print("Distância de edição: {}".format(string_distance(str1, str2)))
\end{minted}
\end{document}
