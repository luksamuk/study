% Created 2022-03-13 dom 21:49
% Intended LaTeX compiler: pdflatex
\documentclass[article, a4paper, oneside, 11pt, english, brazil, sumario=tradicional]{abntex2}
		  \usepackage{times}
\usepackage[utf8x]{inputenc}
\usepackage[T1]{fontenc}
\usepackage{titlesec}
\usepackage[english, hyperpageref]{backref}
\usepackage{hyperref}
\usepackage{indentfirst}
\usepackage{titling}
\usepackage{graphicx}
\ifthenelse{\equal{\ABNTEXisarticle}{true}}{%
\renewcommand{\maketitlehookb}{}
}{}
\titleformat{\section}{\normalfont\normalsize\bfseries\uppercase}{\thesection\space\space}{0pt}{}
\titleformat{\subsection}{\normalfont\normalsize\bfseries}{\thesubsection\space\space}{0pt}{\space}
\titleformat{\subsubsection}{\normalfont\normalsize\bfseries}{\thesubsubsection\space\space}{0pt}{\space}
\titleformat{\paragraph}{\normalfont\normalsize\itshape}{}{0pt}{\theparagraph\space\space}
\setlength{\parindent}{1.5cm}
\setlrmarginsandblock{3cm}{2cm}{*}
\setulmarginsandblock{2.5cm}{2.5cm}{*}
\checkandfixthelayout



\usepackage{minted}
\author{Lucas Samuel Vieira}
\date{\today}
\title{Resumo Individual da Unidade 4}
\hypersetup{
 pdfauthor={Lucas Samuel Vieira},
 pdftitle={Resumo Individual da Unidade 4},
 pdfkeywords={},
 pdfsubject={},
 pdfcreator={Emacs 27.2 (Org mode 9.5.2)}, 
 pdflang={Brazilian}}
\begin{document}

\OnehalfSpacing
\pretextual
\textual

\begin{center}
\textbf{UFVJM - UNIVERSIDADE FEDERAL DOS VALES DO JEQUITINHONHA E MUCURI}

\textbf{BACHARELADO EM SISTEMAS DE INFORMAÇÃO}

\textbf{ENGENHARIA DE SOFTWARE I - SEMESTRE 2021/1}

\textbf{Resumo da Unidade 4 - Engenharia de Requisitos}
\end{center}

\noindent
Nome: Lucas Samuel Vieira
\newline

\section{Resumo}
\label{sec:org5c39eab}

Os  requisitos do  sistema  descrevem o  que o  mesmo  deve fazer,  os
serviços que oferece e as restrições ao funcionamento.

Eles  refletem  a  necessidade  do cliente,  refletem  uma  finalidade
determinada.

O  processo  de  descobrir,  analisar, documentar  e  verificar  esses
serviços é chamado de \textbf{engenharia de requisitos}.

A falha ao  separar \emph{requisitos de usuário} e  \emph{requisitos de sistema}
pode gerar problemas a longo prazo.

\begin{itemize}
\item \textbf{Requisitos  de usuário:}  Declarações  em linguagem  natural e  com
diagramas dos serviços que o  sistema deverá fornecer aos usuários e
restrições com as quais ele deve operar.
\item \textbf{Requisitos de sistema:} Descrições detalhadas das funções, serviços
e restrições  operacionais do  software, definindo exatamente  o que
deve ser implementado; pode ser parte do contrato.
\end{itemize}

Estes tipos de requisitos requerem diferentes tipos de detalhamento, a
depender de seu uso.

\subsection{Requisitos funcionais e não-funcionais}
\label{sec:orgaa5817d}

\begin{itemize}
\item \textbf{Requisitos funcionais:} Declarações de  serviços que o sistema deve
fornecer, como  o sistema deve  ou não  agir para certas  entradas e
como se comportar em certas situações. Geralmente descritos de forma
abstrata e não-técnica.
\item \textbf{Requisitos  não-funcionais:}  Restrições  aos  serviços  e  funções
oferecidos  pelo  sistema,  o  que inclui  \emph{timing},  restrições  no
processo    de   desenvolvimento    e   restrições    impostas   por
normas. Geralmente aplicam-se ao sistema como um todo.
\end{itemize}

\subsection{Documento de requisitos de software}
\label{sec:org29fdbc5}

O  documento é  uma declaração  oficial do  que os  desenvolvedores do
sistema devem implementar, o que inclui tanto os requisitos de usuário
quanto os  requisitos do  sistema. É  essencial quando  um contratante
externo está desenvolvendo o sistema de software.

Para projetos em que os requisitos  sejam instáveis, a confecção de um
documento de requisitos pode ser pouco proveitosa, e poderia dar lugar
a técnicas de desenvolvimento ágil mais propícias.

\subsection{Especificação dos requisitos}
\label{sec:org12c92ca}

A especificação de  requisitos é o processo de  escrever os requisitos
de  usuário  e  de  sistema  em  um  documento  de  requisitos.  Estes
requisitos  devem,  idealmente, ser  claros,  inequívocos  e de  fácil
compreensão, o  que é difícil  de conseguir porque estarão  sujeitos à
interpretação dos \emph{stakeholders}.

Deve-se pensar nos  requisitos do sistema como  versões expandidas dos
requisitos do usuário, sendo utilizados diretamente por engenheiros de
software  como pontos  de  partida  para a  construção  do projeto  do
sistema.  Os  requisitos   de  usuário,  porém,  devem   ter  um  viés
não-técnico,  partindo do  pressuposto da  acessibilidade para  com os
\emph{stakeholders} que não dominem o vocabulário técnico.

\subsection{Processos de engenharia de requisitos}
\label{sec:org0dc6bff}

Estes processos incluem quatro atividades de alto nível:

\begin{enumerate}
\item Estudo de viabilidade -- avalia se o sistema é útil para a empresa;
\item Elicitação e análise -- descoberta de requisitos;
\item Especificação -- conversão dos requisitos em uma forma de padrão;
\item Validação -- avaliação dos  requisitos levantados para com o desejo
do cliente.
\end{enumerate}

Na  prática, porém,  a  engenharia  de requisitos  é  um processo  que
intercala todas essas atividades. Adicionalmente, os esforços mudam de
acordo com a maturidade do projeto; enquanto o esforço inicial está na
compreensão dos requisitos do negócio, os esforços finais estarão mais
concentrados em  elicitar e compreender  em detalhes os  requisitos já
levantados.

\subsection{Elicitação e análise de requisitos}
\label{sec:org4eda909}

Esta  fase ocorre  normalmente  após o  estudo  de viabilidade.  Nela,
engenheiros de software trabalham com  clientes e usuários finais para
compreender o  domínio da  aplicação, os serviços  que o  sistema deve
oferecer, desempenho do sistema, restrições de hardware, etc.

A elicitação é constituida por quatro atividades:

\begin{enumerate}
\item Descoberta de requisitos;
\item Classificação e organização dos requisitos;
\item Priorização e negociação dos requisitos;
\item Especificação dos requisitos.
\end{enumerate}

Nesta fase, destaca-se não apenas  a necessidade de proximidade com os
\emph{stakeholders} e  o domínio  já citados, como  também a  elaboração de
diagramas  de  Caso  de  Uso para  melhor  compreender  os  requisitos
funcionais do sistema.

\subsection{Validação de requisitos}
\label{sec:org210c967}

Nesta fase,  verifica-se se  os requisitos  levantados para  o sistema
realmente  definem o  que o  cliente  espera do  mesmo. Sobrepõe-se  à
análise, já  que, ao invés  de buscar levantar requisitos,  foca-se em
encontrar problemas com os requisitos.

Este passo é importante porque um erro na especificação dos requisitos
pode gerar altos custos e retrabalho posteriormente.

Este processo envolve verificações que incluem:

\begin{enumerate}
\item Verificações de validade;
\item Verificações de consistência;
\item Verificações de completude;
\item Verificações de realismo;
\item Verificabilidade.
\end{enumerate}

Como técnicas a serem utilizadas, pode-se citar:

\begin{enumerate}
\item Revisões de requisitos;
\item Prototipação;
\item Geração de casos de teste.
\end{enumerate}

\subsection{Gerenciamento de requisitos}
\label{sec:org9bb07ac}

Em  geral, requisitos  de um  projeto de  software estão  em constante
mudança, sobretudo em sistemas grandes,  pois estes sistemas tratam de
problemas que não podem ser completamente definidos, o que inviabiliza
a completude dos requisitos.

O gerenciamento de  requisitos é o processo de  compreensão e controle
das mudanças de  requisitos do sistema, o que envolve  manter-se a par
das necessidades individuais dos \emph{stakeholders}  e também a criação de
um processo formal de admissão de propostas de mudanças no software.

Este  processo deve  ser  idealizado  tão logo  se  inicie  a fase  de
elicitação dos requisitos.

\section{Considerações}
\label{sec:org82c8de5}

A engenharia de requisitos permeia  todo o processo de planejamento de
software, sendo provavelmente o ponto mais importante da mesma, já que
uma falha  neste processo pode  implicar em problemas  exponenciais em
outras  fases  da  engenharia  de  software.  Sendo  assim,  parece-me
interessante  gastar o  máximo possível  de esforço  nesta fase,  para
evitar maiores contratempos e também para garantir a alta qualidade de
um software, obviamente impactando na satisfação do cliente final.
\end{document}