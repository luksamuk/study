% Created 2022-03-13 dom 17:48
% Intended LaTeX compiler: pdflatex
\documentclass[article, a4paper, oneside, 11pt, english, brazil, sumario=tradicional]{abntex2}
		  \usepackage{times}
\usepackage[utf8x]{inputenc}
\usepackage[T1]{fontenc}
\usepackage{titlesec}
\usepackage[english, hyperpageref]{backref}
\usepackage{hyperref}
\usepackage{indentfirst}
\usepackage{titling}
\usepackage{graphicx}
\ifthenelse{\equal{\ABNTEXisarticle}{true}}{%
\renewcommand{\maketitlehookb}{}
}{}
\titleformat{\section}{\normalfont\normalsize\bfseries\uppercase}{\thesection\space\space}{0pt}{}
\titleformat{\subsection}{\normalfont\normalsize\bfseries}{\thesubsection\space\space}{0pt}{\space}
\titleformat{\subsubsection}{\normalfont\normalsize\bfseries}{\thesubsubsection\space\space}{0pt}{\space}
\titleformat{\paragraph}{\normalfont\normalsize\itshape}{}{0pt}{\theparagraph\space\space}
\setlength{\parindent}{1.5cm}
\setlrmarginsandblock{3cm}{2cm}{*}
\setulmarginsandblock{2.5cm}{2.5cm}{*}
\checkandfixthelayout



\usepackage{minted}
\author{Lucas Samuel Vieira}
\date{\today}
\title{Resenha Crítica do Vídeo}
\hypersetup{
 pdfauthor={Lucas Samuel Vieira},
 pdftitle={Resenha Crítica do Vídeo},
 pdfkeywords={},
 pdfsubject={},
 pdfcreator={Emacs 27.2 (Org mode 9.5.2)}, 
 pdflang={Brazilian}}
\begin{document}

\OnehalfSpacing
\pretextual
\textual

\begin{center}
\textbf{UFVJM - UNIVERSIDADE FEDERAL DOS VALES DO JEQUITINHONHA E MUCURI}

\textbf{BACHARELADO EM SISTEMAS DE INFORMAÇÃO}

\textbf{ENGENHARIA DE SOFTWARE I - SEMESTRE 2021/1}


\textbf{Resenha Crítica do Vídeo - O que é Arquitetura de Software?}
\end{center}

\noindent
Nome: Lucas Samuel Vieira
\newline

\section{Resumo}
\label{sec:org146e002}

A  questão parece  simples,  mas  é um  dos  tópicos mais  discutidos,
principalmente em grupos de Engenharia de Software.

Existe  uma  diferença  entre  arquitetura de  software  e  design  de
software:

\begin{center}
\textbf{Arquitetura é sempre design. Design nem sempre é arquitetura.}
\end{center}

A   prática   de   Arquitetura   de   Software   busca   maximizar   a
produtividade.  Práticas bem-feitas  aumentam as  entregas, com  menor
esforço.

\textbf{Objetivo:}  Garantir que  os objetivos  de negócio,  os atributos  de
 qualidade e restrições de alto nível sejam atendidos.

Uma boa  arquitetura busca orientar  e propiciar formas  produtivas de
desenvolver, manter, atualizar, entregar e operacionalizar software.

Estas decisões não estarão próximas do código.

Pode-se pensar na arquitetura como:
\begin{itemize}
\item Quais são os componentes que fazem parte do software;
\item Como atribui responsabilidades aos componentes;
\item Como eles se relacionam;
\item Estratégia como padrão coerente para tomada de decisão.
\end{itemize}

Diagrama  de  Contexto,  sugerido  pela C4:  Sistema  ao  centro,  com
aplicações e componentes (sistemas  externos e personas) que induzirão
o sistema a alguma ação ao redor.

As  setas pontilhadas  que vão  dos componentes  para o  sistema dizem
respeito  a relacionamentos,  as relações  que o  software possui  com
aplicações externas.

\begin{center}
\textbf{Arquitetura   de   Software   é    a   o   design   de   componentes, responsabilidades, relacionamentos  e estratégia de  evolução, visando aumentar a  produtividade para desenvolver, manter,  operacionalizar e
distribuir  software,  com o  propósito  de  atender os  objetivos  de negócio de forma efetiva, respeitando e observando as principais restrições e atributos de qualidade.}
\end{center}

\section{Discussão}
\label{sec:org3c8b795}

As ideias do  vídeo me pareceram muito  interessantes sobretudo quando
fala-se  da  questão  dos   componentes  e  de  seus  relacionamentos,
sobretudo    sob   uma    visão   de    microsserviços   e    sistemas
distribuidos.  Como  estes são  temas  em  ascensão na  computação,  é
interessante notar como  a arquitetura de software  pode ser utilizada
para  gerenciar sistemas  de  grande complexidade  desse  tipo. Sem  a
arquitetura de  software, não seria  possível prever e  mitigar muitos
dos problemas que poderiam surgir numa arquitetura de microsserviços.
\end{document}